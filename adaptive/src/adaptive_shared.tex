% SIAM Shared Information Template
% This is information that is shared between the main document and any
% supplement. If no supplement is required, then this information can
% be included directly in the main document.


% Packages and macros go here
\usepackage{lipsum}
\usepackage{amsfonts}
\usepackage{graphicx}
\usepackage{epstopdf}
\usepackage{algorithmic}
\usepackage[T1]{fontenc}
\usepackage[utf8]{inputenc}
\ifpdf
  \DeclareGraphicsExtensions{.eps,.pdf,.png,.jpg}
\else
  \DeclareGraphicsExtensions{.eps}
\fi

% Add a serial/Oxford comma by default.
\newcommand{\creflastconjunction}{, and~}

% Used for creating new theorem and remark environments
\newsiamremark{remark}{Remark}
\newsiamremark{hypothesis}{Hypothesis}
\crefname{hypothesis}{Hypothesis}{Hypotheses}
\newsiamthm{claim}{Claim}

% Sets running headers as well as PDF title and authors
\headers{Adaptive methods using GP for regret-based estimates}{V. Trappler, E. Arnaud, A. Vidard}

% Title. If the supplement option is on, then "Supplementary Material"
% is automatically inserted before the title.
\title{Adaptive methods using GP for regret-based estimates\thanks{Document compiled on \today }% \thanks{Submitted to the editors DATE.
% \funding{This work was funded by the Fog Research Institute under contract no.~FRI-454.}}
}

% Authors: full names plus addresses.
\author{Victor Trappler\thanks{Univ. Grenoble-Alpes  (\email{victor.trappler@univ-grenoble-alpes.fr}, \url{http://vtrappler.github.io/}).}
\and Elise Arnaud \thanks{Univ Grenoble-Alpes} \and Arthur Vidard\footnotemark[3]}

\usepackage{amsopn}
\DeclareMathOperator{\diag}{diag}
\newcommand{\kk}{\theta}
\newcommand{\uu}{u}
\newcommand{\KK}{\theta}
\newcommand{\UU}{U}
\newcommand{\Kspace}{\Theta}
\newcommand{\Uspace}{\mathbb{U}}
\newcommand{\Xspace}{\mathbb{X}}
\newcommand{\Prob}{\mathbb{P}}
\newcommand{\Ex}{\mathbb{E}}
\newcommand{\argmin}{\mathrm{arg}\,\mathrm{min}}
\newcommand{\argmax}{\mathrm{arg}\,\mathrm{max}}
%%% Local Variables: 
%%% mode:latex
%%% TeX-master: "ex_article"
%%% End: 
