\documentclass[a4paper,11pt]{article}
\usepackage[utf8]{inputenc}
\usepackage[english]{babel}
\usepackage{amsmath}
\usepackage{bm}
\usepackage{amsfonts}
\usepackage{graphicx}
\usepackage{geometry}
\usepackage[utf8]{inputenc}
\usepackage{titlesec}
\usepackage{enumitem}
\usepackage{tikz}
\usepackage{calc}
\usepackage{caption}
% \usepackage{zref-xr}
% \externaldocument{article}
\usepackage{easy-todo}
\usepackage{comment}
\usepackage{lipsum}
\usepackage{bbm}
\usepackage{fancyhdr}
\usepackage{tikz}
\usepackage{pdfpages}
\newcommand{\Var}{\mathbb{V}\text{ar}}
\newcommand{\Ex}{\mathbb{E}}
\newcommand{\Prob}{\mathbb{P}}
\DeclareMathOperator*{\argmin}{arg\,min \,}
\DeclareMathOperator*{\argmax}{arg\,max \,}
\usepackage{hyperref}
\usepackage{booktabs}

\begin{document}
\title{JCOMP-D-20-00132: Summary of the changes:\\
  Robust calibration of numerical models based on relative regret
}


\maketitle
\emph{Abstract (p.1)}
\begin{itemize}  
\item l.~5-8: clarified the distinction between control parameters, and uncontrollable random variables
\item l~.11-16: highlighted the role and novelty of the relative regret in this article
\end{itemize}
\emph{Introduction (p.1)}
\begin{itemize}
\item l.~34-37: added source of aleatoric uncertainties for ocean modelling
\item l.~38-48: added effect of omitting the uncertainties in problems
\item l.~48-51: added examples of the necessity of taking into account uncertainties, from a risk management perspective
\item l.~52-56: added different vocabulary found in the literature which convey the same idea
\item l.~57: fixed some typos and english mistakes.
\item l.~60: we explicit the knowledge of the distribution of $\mathbf{U}$.
\item l.~64: added state-of-the-art strategies to model aleatoric uncertainties and citation to Ning \& You.
\item l~.66-68: we clarify that the distributinto $\mathbf{U}$ is not dependent on the value of $\mathbf{k}$
\item l.~73: a reference to an exemple of such a cost function is added
\item l.~82: fixed typo
\item l.~84-85: added the mention of distributional robustness 
\item l.~92: added mention of overcalibration
\item l.~83: typo
\item l.~96,98,99: reformulation for clarity
\item l.~109: added rationale of the relative regret
\item l.~115: typo
\end{itemize}
\emph{Section 2 (p.5)}
\begin{itemize}
\item l.~119: change for clarification
\end{itemize}
\emph{Section 3.1 (p.11)}
\begin{itemize}
\item l.~232-238: clarification of the construction and definition of the relative-regret 
\item l.~253: added reminder that $\mathbf{K}^*$ is discrete.
\item l.~261-272: comparison between additive and relative regret, and justification of the use of relative-regret
\item l.~278-279: formulation of the problem in terms of chance-constraints in Eq.~(20)
\item l.~281-282: clarification of the link with the Value-at-Risk 
\item l.~283-286: interpretation of $\alpha_p$ in terms of bounding probabilistically the relative-error
\item l.~287-288: rewrote the definition of the RRE
\item l.~289-291: typo and english reformation
\item Figure 4 (p.14): changed linestyle, and added quick interpretation in the caption 
\item l.~293: added small synthesis on the properties of the RRE
\end{itemize}
\emph{Section 3.2 (p.14)}
  \begin{itemize}
\item l.~301: added subsection Almost-surely bounded relative-regret
\item l.~302-308: formulation of the a.s.\ bounded relative-regret in terms of chance constraint in Eq.~(21) and property verified by $\mathbf{k}_1$ and $\alpha_1$
\item l.~309-314: link with uncertain set and Savage's minimax approach for decision under uncertainty
\item l.~315-317: SAA formulation of the a.s. bounded constraint
\item l.~317-325: Probabilistic bounding of $\alpha_1$, and level $p=(\eta/2)^{1/N}$ guaranteed probabilistically
\item l.~330-333: clarify formulation of the general case $p \leq 1$
\item l~.336-337: estimation of $\Gamma_\alpha$ with indicator functions (Eq.~(27))
\item l~.339: possibility of unsatisfactory $(p, \alpha_p)$
\item l.~344: minimization of the quantile of order $p$ of the ratio $J / J^*$.
\end{itemize}
 \emph{Conclusion (p.31)}
\begin{itemize}
\item l~.557-564: summary of the robustness aspect of the RRE
\item l~.578-580: reformulation of the sequential design of experiment strategies.
\end{itemize}
\end{document}



%%% Local Variables:
%%% mode: latex
%%% TeX-master: t
%%% End:
